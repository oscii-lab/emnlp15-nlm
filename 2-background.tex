\section{Background}
\label{sec:background}

\subsection{Neural Machine Translation}
Neural machine translation describes approaches to machine translation that
learn from corpora in a single integrated model that embeds words and sentences
into a vector space
\newcite{Kalchbrenner13,journals/corr/ChoMGBSB14,Sutskever14}. We focus on one
recent approach to neural machine translation predicts both a translation and
its alignment to the source sentence \newcite{journals/corr/BahdanauCB14},
though our technique is relevant to related approaches as well.

The architecture consists of an encoder and a decoder. The encoder receives a
source sentence $\mathbf{x}$ and encodes each prefix using a recurrent neural
network that recursively combines embeddings $x_j$ for each word position $j$:
\begin{equation}
\overrightarrow{h}_{j} = f(x_{j}, \overrightarrow{h}_{j-1})
\end{equation}
where $f$ is a non-linear function. Reverse encodings $\overleftarrow{h}_j$ are
computed similarly to represent suffixes of the sentence. These vector
representations are stacked to form $h_j$, a representation of the whole
sentence focused on position $j$.

The decoder predicts each target symbol $\mathbf{y_i}$ sequentially according
to the distribution
\begin{equation}
P(y_{i} | y_{i-1}, ..., y_{1}, \mathbf{x}) = g(y_{i-1}, s_i, c_i)
\label{eqn:normalize}
\end{equation}
where $s_i$ is a hidden decoder state, $c_i$ is a \emph{summary} of the entire
input sequence, and $g$ is another non-linear function. Encoder and decoder
parameters are jointly optimized to maximize the log likelihood of a training
corpus.

Depending on the approach to neural translation, $c$ can take multiple forms.
\newcite{journals/corr/BahdanauCB14} propose integrating an attention mechanism
in the decoder, which is trained to determine on which portions of the source
sentence to focus. The decoder computes $c_{i}$, the summarizing context
vector, as a convex combination of the $h_{j}$. The coefficients of this
combination are proportional (softmax) to an alignment model prediction $\exp a(h_j,
s_i)$, where $a$ is a non-linear function.

The speed of prediction scales with the output vocabulary size, due to the
denominator of Equation~\ref{eqn:normalize} \cite{journals/corr/JeanCMB14}. The
input vocabulary size is also a challenge for storage and learning. As aa
result, neural machine translation systems only consider the top 30K to 100K
most frequent words in a training corpus, replacing the other words with an
\emph{unknown word} symbol.

\subsection{Related Work}
There has been much recent work in improving translation quality with a large
vocabulary size. \newcite{journals/corr/LuongSLVZ14} describe an approach that,
similar to ours, treats the translation system as a black box. They eliminate
unknown symbols by training the system to recognize from where in the source
text each unknown word in the target text came, so that in a postprocessing
phase, the unknown word can be replaced by a dictionary lookup of the
corresponding source word. In contrast, our method does not rely on access to a
complete dictionary, and instead transforms the data to allow the system itself
to learn translations for even the rare words.

Some approaches have altered the model to circumvent the expensive
normalization computation, rather than applying preprocessing and
postprocessing on the text. \newcite{journals/corr/JeanCMB14} develop an
importance sampling strategy for approximating the softmax computation.
\newcite{NIPS2013_5165} present a technique for approximation of the target
word probability using noise-contrastive estimation. Our method can be applied
in conjunction with these techniques.

Sequential or hierarchical encodings of large vocabularies have played an
important role in recurrent neural network language models, primarily to
address the inference time issue of large vocabularies. \newcite{5947611}
describe an architecture in which output word types are grouped into classes by
frequency: the network first predicts a class, then a word in that
class. \newcite{journals/corr/abs-1301-3781} describe an encoding of the output
vocabulary as a binary tree. To our knowledge, hierarchical encodings have not
been applied to input vocabulary encoding for translation.

Other methods have also been developed to work around large-vocabulary issues
in language modeling. \newcite{Morin05hierarchicalprobabilistic},
\newcite{NIPS2008_3583}, and \newcite{strategies} develop hierarchical versions
of the softmax computation; \newcite{Huang:2012:IWR:2390524.2390645} and
\newcite{Collobert:2008:UAN:1390156.1390177} remove the need for normalization,
thus avoiding computation of the summation term over the entire vocabulary.

\subsection{Huffman Codes}

An encoding can be used to represent a sequence of tokens from a large
vocabulary $\mathcal{V}$ using a small vocabulary $\mathcal{W}$.  In the case
of translation, let $\mathcal{V}$ be the original corpus vocabulary, which can
number in the millions of word types in a typical corpus. Let $\mathcal{W}$ be
the vocabulary size of a neural translation model, typically
set to a much smaller number such as 30,000.

A deterministically invertible, variable-length encoding maps each
$v\in\mathcal{V}$ to a sequence $w \in \mathcal{W}+$ such that no other
$v'\in\mathcal{V}$ is mapped to a prefix of $w$. Encoding simply replaces each
element of $\mathcal{V}$ according to the map, and decoding is unambiguous because of this
prefix restriction. An encoding can be represented as a tree in which each leaf
corresponds to an element of $\mathcal{V}$, each node contains a symbol from
$\mathcal{W}$, and the encoding of any leaf is its path from the root.

A Huffman code is an optimal encoding that uses as few symbols from
$\mathcal{W}$ as possible to encode an original sequence of symbols from
$\mathcal{V}$. Although binary codes are typical, $\mathcal{W}$ can have any
size. An optimal encoding can be found using a greedy algorithm
\cite{huffman}.

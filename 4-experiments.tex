\section{Experimental Results}
\label{sec:exp}

We train the public implementation of the system described in
\newcite{journals/corr/BahdanauCB14} on the English-French parallel corpora
from ACL WMT '14, which contains 348M tokens. We evaluate on news-test-2014,
also from WMT '14, which contains 3003 sentences. All experiments use the same
learning parameters and vocabulary size of 30,000.

We construct each encoding by the following method. First, we use the formulas
derived in the previous section to calculate the optimal number of common words
$W$ for each encoding scheme, using $V$ to be the true vocabulary size of the
training corpus and $W = 30,000$. We then find the $C$ most common words in the
text and encode them as themselves. For the remaining rare words, we encode
them using a distinct symbol whose form matches the one prescribed for each
encoding scheme. This encoding scheme is applied separately to both the source
text and the target text.

To construct the mapping from rare words to their 2-word encodings, we bin rare
words by frequency into branches. Thus, rare words of similar frequency in the
training corpus tend to have encodings with the same first symbol. Similarly, the
standard Huffman construction algorithm groups together rare words with similar
frequencies within subtrees. More intelligent heuristics for constructing trees,
such as using translation statistics instead of training corpus frequency, will be
an interesting area of future work.

We use the RNNsearch-50 architecture from \newcite{journals/corr/BahdanauCB14}
as our machine translation system. We report results for this system alone, as
well as for each of our three encoding schemes, using the BLEU metric
\cite{bleu}. Our results after training each variant for 5 days, corresponding to
roughly 2 passes through the 180K-sentence training corpus, appear in
Table~\ref{table:results}.

Our results indicate that the scheme that keeps the highest number of common
words, \emph{Repeat-All}, performs best. Table~\ref{table:precs} shows the
unigram precision of each output. The common word translation accuracy does not
change significantly among strategies, but encoding rare words using both
pseudo-words and common words gives substantially higher rare word accuracy.

It is possible for the model to predict an invalid encoded sequence that does
not correspond to anything in the original vocabulary. However, in our experiments,
we did not observe any invalid sequences in the decoding of the test set. A reasonable
way to deal with invalid sequences would be to drop them from the output during decoding.

\begin{table}
  \centering
  \vspace{8pt}
  \tabcolsep=0.11cm{
  \begin{tabular}{cccc}
    \toprule
      \textbf{Encoding} & \textbf{BLEU} & \textbf{\# Common Words}\\
    \midrule
      None & 25.77 & 30,000\\
    \midrule
      Repeat-All & 27.45 & 29,940\\
    \midrule
      Repeat-Symbol & 26.52 & 28,860\\
    \midrule
      No-Repeats & 25.79 & 27,320\\
    \bottomrule
  \end{tabular}}
  \caption{BLEU scores (\%) on detokenized test set for each encoding
    scheme after training for 5 days.}
  \label{table:results}
\end{table}

\begin{table}
  \centering
  \vspace{8pt}
  \tabcolsep=0.11cm{
  \begin{tabular}{cccc}
    \toprule
      \textbf{Encoding} & \textbf{Common} & \textbf{Rare}\\
    \midrule
      None & 62.0 & 0.0 \\
    \midrule
      Repeat-All & 65.8 & 28.0 \\
    \midrule
      Repeat-Symbol & 65.5 & 16.5 \\
    \midrule
      No-Repeats & 63.6 & 13.2 \\
    \bottomrule
  \end{tabular}}
  \caption{Test set precision (\%) on common words and rare words for each encoding strategy.}
  \label{table:precs}
\end{table}

\section{Introduction}
\label{sec:intro}

\newcite{journals/corr/BahdanauCB14} propose a neural translation model
that learns vector representations for individual words as well as word
sequences. Their approach jointly predicts a translation and a latent
word-level alignment for a sequence of source words. However, the architecture
of the network does not scale naturally to large vocabularies
\cite{journals/corr/JeanCMB14}.

In this paper, we propose a novel approach to circumvent the large-vocabulary
challenge by preprocessing the source and target word sequences, encoding them
as a longer token sequence drawn from a small vocabulary that does not
discard any information. Common words are unaffected, but rare words are
encoded as a sequence of two pseudo-words. The exact same learning and
inference machinery applied to these transformed data yields improved
translations.

We evaluate a family of 3 different encoding schemes based on Huffman codes.
All of them  eliminate the need to replace rare words with the
\emph{unknown word} symbol. Our approach is simpler than other methods recently
proposed to address the same issue. It does not introduce new parameters into
the model, change the model structure, affect inference, require access to a
complete dictionary, or require any additional learning procedures.
Nonetheless, compared to a baseline system that replaces all rare words with an
\emph{unknown word} symbol, our encoding approach improves English-French
news translation by up to 1.7 BLEU.

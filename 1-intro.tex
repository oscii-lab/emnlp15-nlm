\section{Introduction}
\label{sec:intro}

\newcite{journals/corr/BahdanauCB14} propose a deep learning approach to
machine translation that learns vector representations for individual words as
well as word sequences. Their approach jointly predicts a translation and a
latent word-level alignment for a sequence of source words. However, the architecture of
the network poses a challenge for predicting rare words in large
vocabularies: the task of estimating the vector representations of many words
is both resource-constrained and information-constrained: storage space and model learning time
are increased. In this paper, we
propose a novel approach to circumvent the large-vocabulary challenge by preprocessing the source
and target word sequences, transforming them to an alternative token sequence drawn from
relatively smaller vocabularies. Words of high frequency encode to themselves, and rare
words encode to a sequence of two pseudo-words. As a result, the exact same learning machinery
can yield improved translations without imposing limits on the source or target
vocabulary size.

Our approach considers a family of 3 different encoding schemes, eliminating the need to replace every rare word with the
\emph{unknown word} symbol. The desirability of removing all occurrences of this symbol is evident: translation
quality otherwise significantly deteriorates for sentences containing many such words, or for
languages with naturally large vocabularies. Our approach additionally benefits from simplicity over
other methods developed to address the same issue. It does not introduce
new parameters into the model or change what the system learns, nor does it require access to a complete dictionary.
 As compared to the original system, which does use the \emph{unknown word} symbol in
its translations, our methods achieve an improvement in English-French translation of up to 6.52\% BLEU, or about 1.7 BLEU points,
and can easily be used in conjunction with other techniques.